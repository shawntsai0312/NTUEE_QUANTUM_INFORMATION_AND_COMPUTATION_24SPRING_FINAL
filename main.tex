%% Please do not delete the following line
%% This is the Overleaf LaTeX template for the journal Nuclear Physics A.
%% Copyright 2007-2020 Elsevier Ltd
%% 
%% This file is part of the 'Elsarticle Bundle'.
%% ---------------------------------------------
%% 
%% It may be distributed under the conditions of the LaTeX Project Public
%% License, either version 1.2 of this license or (at your option) any
%% later version.  The latest version of this license is in
%%    http://www.latex-project.org/lppl.txt
%% and version 1.2 or later is part of all distributions of LaTeX
%% version 1999/12/01 or later.
%% 
%% The list of all files belonging to the 'Elsarticle Bundle' is
%% given in the file `manifest.txt`''.
%% 
%% Template article for Elsevier's document class `elsarticle'
%% with harvard style bibliographic references

%\documentclass[preprint,12pt,authoryear]{elsarticle}

%% Use the option review to obtain double line spacing
%% \documentclass[authoryear,preprint,review,12pt]{elsarticle}

%% Use the options 1p,twocolumn; 3p; 3p,twocolumn; 5p; or 5p,twocolumn
%% for a journal layout:
%% \documentclass[final,1p,times,authoryear]{elsarticle}
%% \documentclass[final,1p,times,twocolumn,authoryear]{elsarticle}
%% \documentclass[final,3p,times,authoryear]{elsarticle}
%% \documentclass[final,3p,times,twocolumn,authoryear]{elsarticle}
%% \documentclass[final,5p,times,authoryear]{elsarticle}
\documentclass[final,5p,times,twocolumn,authoryear]{elsarticle}

%% For including figures, graphicx.sty has been loaded in
%% elsarticle.cls. If you prefer to use the old commands
%% please give \usepackage{epsfig}

%% The amssymb package provides various useful mathematical symbols
\usepackage{amssymb}
\usepackage{lipsum}
%% The amsthm package provides extended theorem environments
%% \usepackage{amsthm}

%% The lineno packages adds line numbers. Start line numbering with
%% \begin{linenumbers}, end it with \end{linenumbers}. Or switch it on
%% for the whole article with \linenumbers.
%% \usepackage{lineno}

%% You might want to define your own abbreviated commands for common used terms, e.g.:
\journal{Nuclear Physics A}


\begin{document}

\begin{frontmatter}

\title{Surface code related paper survey}

\author{Group 28}

\begin{abstract}
%% Text of abstract
Today, Quantum computers have high error rates compared to the classical computer. In order to make quantum computers useful, error rates have to be as low as 1 in a trillion. While a typical transistor in a microprocessor can run for about a billion years at a billion operations per second, without ever suffering a hardware fault due to any form of interference. A huge improvement in performance is needed, since the typical quantum bits become randomized in about one one-thousandth of a second. Classical error correction employ redundacy, however, it is impossible for quantum code due to no-cloning theorem. In this review, we will summarize one of the promising method – surface code. Besides, the novel improvement of the algorithm and decoder will also be provided.
\end{abstract}

%%Graphical abstract
%\begin{graphicalabstract}
%\includegraphics{grabs}
%\end{graphicalabstract}

%%Research highlights
%\begin{highlights}
%\item Research highlight 1
%\item Research highlight 2
%\end{highlights}

\begin{keyword}
%% keywords here, in the form: keyword \sep keyword, up to a maximum of 6 keywords
Quantum Computation \sep Quantum Error Correction \sep Surface Code \sep Stabilizer codes

%% PACS codes here, in the form: \PACS code \sep code

%% MSC codes here, in the form: \MSC code \sep code
%% or \MSC[2008] code \sep code (2000 is the default)

\end{keyword}


\end{frontmatter}

%\tableofcontents

%% \linenumbers

%% main text

\section{Introduction}
\label{introduction}

Here is where you provide an introduction to work and some background. For example building on previous work of image enhancement in optical astronomy \citep{vojtekova2021learning}, \cite{sweere2022deep} developed a method to improve the resolution of X-ray images from XMM-Newton to provide similar spatial resolution to Chandra.

\section{Title 2}
%%\label{}
\lipsum[1]

\subsection{Subsection title}

A random equation, the Toomre stability criterion:

\begin{equation}
    Q = \frac{\sigma_v \times \kappa}{\pi \times G \times \Sigma}
\end{equation}

\section{Title 3}
%%\label{}
\lipsum[2]

\subsection{Subsection title}
\lipsum[3]

% \begin{table}
% \begin{tabular}{l c c c} 
%  \hline
%  Source & RA (J2000) & DEC (J2000) & $V_{\rm sys}$ \\ 
%         & [h,m,s]    & [o,','']    & \kms          \\
%  \hline
%  NGC\,253 & 	00:47:33.120 & -25:17:17.59 & $235 \pm 1$ \\ 
%  M\,82 & 09:55:52.725, & +69:40:45.78 & $269 \pm 2$ 	 \\ 
%  \hline
% \end{tabular}
% \caption{Random table with galaxies coordinates and velocities, Number the tables consecutively in
% accordance with their appearance in the text and place any table notes below the table body. Please avoid using vertical rules and shading in table cells.
% }
% \label{Table1}
% \end{table}


\section{Discussion}
%%\label{}
\lipsum[4]

\section{Summary and conclusions}
%%\label{}
\lipsum[1-4]


\section*{Acknowledgements}
Thanks to ...

%% The Appendices part is started with the command \appendix;
%% appendix sections are then done as normal sections
\appendix

\section{Appendix title 1}
%% \label{}

\section{Appendix title 2}
%% \label{}

%% If you have bibdatabase file and want bibtex to generate the
%% bibitems, please use
%%
\bibliographystyle{elsarticle-harv} 
\bibliography{example}

%% else use the following coding to input the bibitems directly in the
%% TeX file.

%%\begin{thebibliography}{00}

%% \bibitem[Author(year)]{label}
%% For example:

%% \bibitem[Aladro et al.(2015)]{Aladro15} Aladro, R., Martín, S., Riquelme, D., et al. 2015, \aas, 579, A101


%%\end{thebibliography}

\end{document}

\endinput
%%
%% End of file `elsarticle-template-harv.tex'.

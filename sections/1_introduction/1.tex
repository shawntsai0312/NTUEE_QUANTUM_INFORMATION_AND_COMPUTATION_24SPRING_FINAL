Quantum computing is a rapidly developing field, and its future is full of exciting possibilities. With breakthroughs from various promising teams, quantum computers are fast approaching the point where they can start to benefit our society. However, in order to fully explore the potential of quantum computers, complex design flows have to be carefully designed and employed to boost the accuracy and efficiency of quantum computing development. Among the different methodologies, error correction is a critical component in ensuring the reliable operation of quantum computers. One of the most prominent error correction schemes is the surface code, which has garnered significant attention due to its robustness and scalability. Surface codes, a class of topological quantum codes, play a vital role in protecting quantum information from decoherence and operational errors. The surface code utilizes a two-dimensional lattice of qubits with stabilizers to detect and correct errors. Recent advancements have focused on improving the efficiency and performance of surface code decoders, stabilizers, and associated algorithms, which are pivotal for practical quantum computing. Generally, implementing and improving surface codes is a complex task. Unlike classical error correction, quantum error correction deals with qubits that exist in superpositions, making the detection and correction of errors more challenging. The design and optimization of decoders and stabilizers, which are responsible for identifying and rectifying errors, are crucial for enhancing the fidelity of quantum computations.

\textbf{to be modified}
In Section II, we provide a comprehensive review of the surface code, including its foundational principles and its importance in quantum error correction. In Section III, we delve into recent improvements in surface code decoders, examining various approaches to enhance their accuracy and efficiency. Section IV discusses advancements in stabilizer codes and their role in the implementation of the surface code. Finally, in Section V, we explore algorithmic developments that leverage the surface code, highlighting their potential impact on the future of quantum computing.
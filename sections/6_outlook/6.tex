In this report, we have dived into the pivotal aspects of quantum error correction, emphasizing the principles and recent developments of the surface code. We started by providing a review of the surface code, detailing its foundational principles and significance in enhancing the reliability of quantum computations. If readers want to know more about the principles of the surface code, they are more than welcome to explore the extensive resources and references provided.

We then examined recent advancements in the surface code, reviewing papers about advanced decoding techniques, which achieves more efficient and accurate correction cycles. Following is a section where we discuss special designs of stabilizers to let surface codes have the ability to deal with defective latticies. Furthermore, we reported on some successful implementations of the surface code in very recent years, underscoring significant milestones in the journey towards realizing large-scale quantum computers.

Quantum computation is an exciting topic in recent years for its paradigm-shifting potential to revolutionize various fields, from cryptography to drug discovery, by harnessing the unique principles of quantum mechanics. However, before we can fully develop this potential, fields surrounding quantum computation, such as quantum algorithms, error correction, and hardware development, require significant advancements. Ultimately, these advancements will pave the way for realizing the transformative promise of quantum computation and lead us to a new era of innovation and discovery.